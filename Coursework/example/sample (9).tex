 \documentclass{article}
\usepackage{amssymb}
\usepackage{srcltx}
\usepackage{amsmath}
\usepackage[14pt]{extsizes}
\usepackage[mag=1000,a4paper,   	
    left=2cm , right=2cm, top=2cm, bottom=2cm,
	]{geometry}
\usepackage[sort,compress]{cite}
\usepackage[english]{babel}

\begin{document}


\title {One approach to fault dictionary size
reduction}


\maketitle

\begin{abstract}
    This paper describes  an approach to reduction of a diagnostic
    information with a help of masks of fault dictionaries. The
    new algorithm for finding of a mask is proposed. The new
    algorithm has appropriate time characteristics and small
    memory requirements. The experimental results show an
    effectiveness of proposed algorithm on  fault dictionaries of
    circuits in ISCAS'89 benchmark set.

    {\it Keywords:} Diagnosis, fault dictionary, diagnostic
    information reduction.

    \emph{SS3: Dependability and Testing of Digital Systems}

    \emph{SoC and NoC Dependability and Testing}

\end{abstract}


\section{Introduction}

Fault dictionaries are the  most widely  used technique for
testing circuits. Pre-computed fault dictionaries stores the
signature of each fault under a fault model. Diagnosis using a
fault dictionary compares a signature derived from the circuit
under test to pre-stored ones. Pre-stored fault in the dictionary
most similar to the considered one will be the final conclusion.

The simplest fault signature that can be obtained is a full
response of circuit to a test set in the presence of the fault
under a fault model. The \emph{full response dictionary} is a
fault dictionary that consists of such signatures.

The main objection to the usage of full response dictionaries
consists of their extremely large size. Such dictionary can
require Gbits of storage for today's integrated circuits.

But often dictionaries  contain redundant information and can be
made significantly smaller \cite{Ryan-Fuchs-Pomeranz, Speran-book,
Boppana-HF, Boppana-F, Arslan-Orailoglu, Pomeranz-Reddy, Chess-L}.
In order to reduce the space needed to store this data set, many
techniques have been proposed.

Some of the techniques for dictionary size reduction is based on
using of so-called ``masks'' \cite{Speran-book, Chipulis-75-1,
Chipulis-75-2, Chipulis-77, Malyshenko-Razdobreev, Sharshunov-73,
Speranskiy-84, Speranskiy-Shatokhina-85}, when fault dictionary
stores ``masked'' fault reaction to the test sequence as a fault
signature.

The mask for full response dictionary can be represented as a
bit-vector, where each bit corresponds to the position of full
reaction of circuit to the test sequence. Zero bits of the mask
show the redundant positions of the full reaction. The
output-compacted signature of a fault is the result of application
of the mask to the full response of circuit to a test set in the
presence of the fault under a fault model. It consists of  values
taken from the positions of full reaction corresponding to nonzero
positions of the mask. Therefore, a number of nonzero bits in the
mask is responsible for reduction of the fault dictionary size.

Diagnosis using a mask extracts a fault signature from full
response of a circuit under test and compares it with pre-computed
signatures in the fault dictionary.

Fault dictionary compression with use of mask may be essential in
the other techniques of dictionary size reduction.

A \emph{pass-fail dictionary} is a matrix where one bit of
information per pair (fault, test vector) stored~--- a 1 if the
circuit will fail the test in the presence of fault and 0 if it
will pass. A diagnostic resolution of pass-fail dictionary may be
worse than the diagnostic resolution of full response dictionary,
but its size at least $r$ times less than the size of full
response dictionary, where $r$ is the number of circuit primary
outputs.

In practice the size of pass-fail dictionary can be reduced with
the help of mask. In this case the records for each fault in
pass-fail dictionary are used as the initial data for method of
finding of a mask.

Pomeranz and Reddy's \cite{Pomeranz-Reddy} proposed to use a
greedy algorithm to choose columns from the full response
dictionary to augment a pass-fail dictionary. The aim is to get
\emph{Compact dictionary} with little or no erosion of diagnostic
resolution for the modelled faults. The mentioned greedy algorithm
is, as the matter of fact, an algorithm of finding of a mask.

Arslan and Orailoglu \cite{Arslan-Orailoglu} offered a method
where test vector set partitioned into blocks and a fault
dictionary stores a combined signature for each test vector
partition instead of pass-fail information for each test vector.
The second phase of this method consist of finding of a mask for
the XORed pass-fail dictionary obtained after the first phase.

Boppana, Hartanto and Fuchs \cite{Boppana-HF} proposed a
dictionary organization where entry for each fault contain a
delimited list of only faulty primary outputs for each test
vector. Such dictionary can be more compact if it is created from
a pre-found masked full-response dictionary, i.~e. if it does not
include excess faulty outputs.

The ideal mask provides a maximal reduction of a fault dictionary
volume with the best diagnostic resolution. The problem of finding
of such mask is NP hard and implies memory- and time-consuming
computations.

Initial data for finding of a mask is fault dictionary (full
response dictionary or pass-fail dictionary). Originally, for even
moderate volume of  fault dictionary the problem of finding a mask
must be solved with the aid of suboptimal heuristic methods.

In this work we offer a new algorithm of this type. This
algorithm, by definition, do not ensure that the result is
optimal, but algorithm's result can be close enough to ideal.
Moreover the timing data of algorithm and its memory requirements
can be reckoned in advantage of its using.


\section{Preliminaries}
Further we are using following notations.

Consider a  circuit with $N$ modelled faults
$\{f_1,f_2,\dots,f_N\}$. Denote $F=\{f_i\mid 1 \leqslant
i\leqslant N\}$. Let's the fault dictionary is  the matrix
$D=\|d_{ij}\|_{N\times M}$, where $i$-th row is a signature of
$i$-th modelled fault.

Let $R$ is the number of pairs of distinguishable rows in $D$ and
$P=N(N-1)/2$ is  the total number of row pairs.  We use the value
$R/P$ as the diagnostic resolution of fault dictionary $D$.

Any mask $H$ for dictionary $D$ is a bit-vector of length $M$. We
define a volume of mask $H$ as a number of its non-zero bits. The
result of application of mask $H$ to a $D$ is a new fault
dictionary $D_{H}$. Matrix $D_H$ is received from a matrix $D$ by
removal of the columns corresponding to zero bits in mask $H$.

In general, we can state the problem of finding mask as following:
for given fault dictionary $D$ it is necessary to find mask $H$ of
minimal volume preserving the diagnostic resolution of $D$ for
$D_H$. In other words  it is necessary to find such set of columns
of matrix $D$, that provides the same number of pairs of
distinguishable rows.

%When the aim of dictionary reduction is the dictionary of some
%particular size it can be stated as following: for given fault
%dictionary $D$ it is necessary to find a mask $H$ of given volume
%providing maximal diagnostic resolution.


\section{Algorithm}
Each column of $D$ partitions set of faults $F$ into subsets
according to values in this column. The collection of these
subsets can be further partitioned by the other columns of $D$. If
diagnostic resolution of $D$ is equal to 1 then set $F$
consequently partitioned by each columns of $D$ will lead to
singleton set.

In this terms, to find a mask of $D$ means to find a subset of
columns of $D$, that provide the same partitioning of $F$ as $D$
do.

To choose a columns for best partitioning we use the following
ideas.

Let we have the set of faults $F'\subseteq F$ and $|F'|$ is the
cardinality of $F'$. Then value
\begin{equation}\label{eq:01}
    I(F')=\log_2|F'|
\end{equation}
indicates an average quantity of information needed for the
identification of a fault from set $F'$. When $F'$ is partitioned
into $\{F'_1, \ldots, F'_{z_k}\}$ according to the values of
$k$-th column the expression
\begin{equation}\label{eq:02}
    I_k(F')=\sum\limits_{j=1}^{z_k}\frac {|F'_j|}{|F'|}\cdot I(F'_j)
\end{equation}
indicates the quantity of information needed for the
identification after the partition.

Therefore,  for the best partitioning it is necessary to choose a
column which delivers the maximal information gain
\begin{equation}\label{eq:03}
    G(k,F')=I(F')-I_k(F').
\end{equation}

As after the first partition of $F$ we have a collection of sets
it would be more preferably to choose such column of a matrix $D$
which would deliver the best partitioning for the whole
collection.

Let $\widetilde F$ is a collection of subsets of $F$. Then value
\begin{equation}\label{eq:04}
    I'(\widetilde F)=\sum\limits_{F'\in \widetilde F}\frac
    {|F'|}{|F|}\cdot I(F')
\end{equation}
is an average quantity of information still needed for the
identification of any fault from $F$. When collection $\widetilde
F$ is partitioned according to the values of $k$-th column the
value
\begin{equation}\label{eq:05}
    I_k(\widetilde F)=\sum\limits_{F'\in \widetilde F}\frac
    {|F'|}{|F|}\cdot I_k(F')
\end{equation}
is the quantity of information needed for the identification after
the partition.

In this terms for the best result it is necessary to choose a
column which provides the maximal information gain
\begin{equation}\label{eq:06}
    G'(k,\widetilde F)=I'(\widetilde F)-I'_k(\widetilde F).
\end{equation}

If $\widetilde F$ consist of sets of undistinguishable faults then
$G'(k,\widetilde F)=0$ for any $k$. This fact may be accepted as a
termination criteria for partitioning.



\section{Experimental results}
The proposed algorithm  has been applied to fault dictionaries of
circuits in ISCAS'89 benchmark set. For each benchmark circuit we
created a maximal resolution full response dictionary for HITEC
test vectors \cite{HITEC} and single stuck-at faults. We used PC
with Celeron 1700 MHz processor and 256 MB of RAM.



\section{Conclusion}

In this paper one approach to fault dictionary reduction with the
help of masks is proposed. A new algorithm for finding of a mask
of a fault dictionary is presented. The given results show that
the algorithm finds acceptable masks in short time and with small
memory requirements.

The sets of experiments on the ISCAS'89 benchmark circuits
demonstrate the significant dictionary reduction using masks found
with the offered algorithm.

\bibliographystyle{plain}
\bibliography{my}


\end{document}
